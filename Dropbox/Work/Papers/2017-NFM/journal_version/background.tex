\section{Background}
\label{sec:background}

\subsection{CPSML Models}
\label{sec:cps}

In this work, we consider models of cyber-physical systems with machine learning components (CPSML).
We assume that a system model is given as a gray-box simulator defined as a tuple $M= (\stsp, \insp, \simu)$,
where $\stsp$ is a set of system states, $\insp$ is a set of input values, and
$\simu: \stsp \times \insp \times \tsp  \to \stsp$ is a simulator that maps 
a state $\vx(t_k) \in \stsp$ and input value $\vu(t_k) \in \insp$ at time $t_k \in \tsp$ to a new 
state $\vx(t_{k+1}) = \simu(\vx(t_k), \vu(t_k),t_k)$, where $t_{k+1} = t_k + \Delta_k$ for a 
time-step $\Delta_k \in \rationals_{> 0}$.

Given an initial time $t_0 \in \tsp$, an initial state $\vx(t_0) \in \stsp$, a sequence of time-steps
$\Delta_0, \dots, \Delta_n \in \rationals_{> 0}$, and a sequence of input values $\vu(t_0), \dots, \vu(t_n) \in \insp$,
a simulation trace of the model $M= (\stsp, \insp, \simu)$ is a sequence:
$$(t_0,\vx(t_0),\vu(t_0)),(t_1,\vx(t_1),\vu(t_1)), \dots, (t_n,\vx(t_n),\vu(t_n))$$
where $\vx(t_{k+1}) = \simu(\vx(t_k),\vu(t_k),\Delta_k)$ and $t_{k+1} = t_{k} + \Delta_k$ for $k = 0,\dots, n$.

The gray-box aspect of the CPSML model is that we assume some knowledge
of the internal ML components. Specifically, these components,
termed \emph{classifiers},
are functions $\class : \featsp \to \labsp$ that assign to their input 
\emph{feature vector} $\feat \in \featsp$ a \emph{label} $\lab \in \labsp$,
where $\featsp$ and $\labsp$ are a feature and label space,
respectively. Without loss of generality, we focus on binary classifiers whose label space is $\labsp = \{0,1\}$.
An ML algorithm selects 
a classifier using a training set $\{(\ifeat{1},\ilab{1}),\dots,(\ifeat{m},\ilab{m})\}$
where the $(\ifeat{i},\ilab{i})$ are labeled examples with $\ifeat{i} \in \featsp$ and
$\ilab{i} \in \labsp$, for $i = 1,\dots, m$.
The quality of a classifier can be estimated on a test set of examples 
comparing the classifier predictions 
against the labels of the examples. Precisely, for a given 
test set $T = \{(\ifeat{1},\ilab{1}),\dots,(\ifeat{l},\ilab{l})\}$,
the number of false positives $\fp{f}{T}$ and false negatives $\fn{f}{T}$ of 
a classifier $f$ on $T$ are defined as:
\begin{equation}
	\begin{split}
		%\tp{f}{T} = \{ \ifeat{i} \in T \mid \class(\ifeat{i}) = 1 \text{ and } \ilab{i} = 1 \} \\
		%\tn{f}{T} = \{ \ifeat{i} \in T \mid \class(\ifeat{i}) = 0 \text{ and } \ilab{i} = 0 \} \\
		\fp{f}{T} = &\ \mid \{ \ifeat{i} \in T \mid \class(\ifeat{i}) = 1 \text{ and } \ilab{i} = 0 \} \mid \\
		\fn{f}{T} = &\ \mid \{ \ifeat{i} \in T \mid \class(\ifeat{i}) = 0 \text{ and } \ilab{i} = 1 \} \mid \\
	\end{split}
\end{equation}
The error rate of $f$ on $T$ is given by:
\begin{equation}
	\err{f}{T} = (\fp{f}{T}+\fn{f}{T}) / l 
\end{equation}
A low error rate implies good predictions of the classifier $f$ on the test 
set $T$.

\subsection{Signal Temporal Logic}

We consider Signal Temporal Logic~\cite{maler2004monitoring}
(STL) as the language to specify properties to be verified against a CPSML model.
STL is an extension of linear temporal logic (LTL) suitable for the 
specification of properties of CPS.

A \emph{signal} is a function $s : D \to S$, with $D \subseteq \reals_{\geq 0}$ 
an interval and either $S \subseteq \bools$ or $S \subseteq \reals$,
where $\bools = \{\top, \bot\}$ and $\reals$ is the set of reals. 
Signals defined on $\bools$ are called \emph{booleans}, while
those on $\reals$ are said \emph{real-valued}. 
A \emph{trace} $w = \{s_1,\dots, s_n\}$ is a finite set of real-valued signals
defined over the same interval $D$.

Let $\Sigma = \{\pred_1, \dots, \pred_k \}$ be a finite set of predicates $\pred_i : \reals^n \to \bools$,
with $\pred_i \equiv p_i(x_1, \dots, x_n) \lhd 0$, $\lhd \in \{ <, \leq \}$, and $p_i : \reals^n \to \reals$
a function in the variables $x_1, \dots, x_n$.

An STL formula is defined by the following grammar:
\begin{equation}
	\varphi := \pred \, | \, \neg \varphi \, | \, 
                    \varphi \wedge \varphi \, | \, \varphi \U{I} \varphi
\end{equation}
where $\pred \in \Sigma$ is a predicate and $I \subset \reals_{\geq 0}$ is a closed
non-singular interval. Other common temporal operators
can be defined as syntactic abbreviations in the usual way, like for instance
$\varphi_1 \vee \varphi_2 := \neg ( \neg \varphi_1 \wedge \varphi_2 )$,
$\F{I} \varphi := \top \U{I} \varphi$, or $\G{I}\varphi := \neg \F{I} \neg \varphi$.
Given a $t \in \reals_{\geq 0}$, a shifted interval $I$
is defined as $t + I = \{t + t' \mid t' \in I\}$.

%The \emph{time horizon} of a formula is the last time instant to which a
%formula refers. %We assume that the time horizon of a trace is greater than one of an evaluated formula.

\begin{definition}[Qualitative semantics]
	Let $w$ be a trace, $t \in \reals_{\geq 0}$, and $\varphi$ be an STL formula.
	The \emph{qualitative semantics} of $\varphi$ is inductively defined as follows:
	\begin{equation}
		\begin{split}
			w,t \models  \pred \text{ iff } 			&\pred(w(t)) \text{ is true} \\
			w,t \models \neg\varphi \text{ iff } 		& w,t \not\models \varphi \\
			w,t \models \varphi_1 \wedge \varphi_2 \text{ iff }	& w,t \models \varphi_1 \text{ and } w,t \models \varphi_2 \\
			w,t \models \varphi_1 \U{I} \varphi_2 \text{ iff }	& \exists t' \in t + I \text{ s.t. } w,t' \models \varphi_2 \text{ and } \forall t'' \in [t,t'], w,t'' \models \varphi_1 \\
		\end{split}
	\end{equation}
\end{definition}

A trace $w$ satisfies a formula $\varphi$ if and only if $w,0 \models \varphi$, in short $w \models \varphi$.
For given signal $w$, time instant $t\in\reals_{\geq 0}$, and STL formula $\varphi$, the \emph{satisfaction
signal} $\satsig{w}{t}{\varphi}$ is $\top$ if $w,t \models \varphi$,  $\bot$ otherwise.
%\begin{equation}
%	\satsig{w}{t}{\varphi} = \begin{cases}
%		\top & \text{ if } w,t \models \varphi \\
%		\bot & \text{ otherwise}\\
%	\end{cases}
%\end{equation}

Given a CPSML model $M = (\stsp, \insp, \simu)$, $M \models \varphi$ if
every simulation trace of $M$ satisfies $\varphi$.

\begin{definition}[Quantitative semantics]
	Let $w$ be a trace, $t \in \reals_{\geq 0}$, and $\varphi$ be an STL formula.
	The \emph{quantitative semantics} of $\varphi$ is defined as follows:
	\begin{equation}
		\begin{split}
			\rob{p(x_1, \dots, x_n) \lhd 0}{w}{t} = &\ p(w(t)) \text{ with } \lhd \in \{<,\leq \} \\
			\rob{\neg\varphi}{w}{t} = &\ -\rob{\varphi}{w}{t} \\
			\rob{\varphi_1 \wedge \varphi_2}{w}{t} = &\ \min( \rob{\varphi_1}{w}{t}, \rob{\varphi_2}{w}{t} ) \\
			\rob{\varphi_1 \U{I} \varphi_2}{w}{t} = &\ \sup_{t' \in t+I} \min( \rob{\varphi_2}{w}{t'}, \inf_{t''[t,t']} \rob{\varphi_1}{w}{t''} ) \\
		\end{split}
	\end{equation}
\end{definition}
The \emph{robustness} of a formula $\varphi$ with respect to a trace $w$ is the signal $\rob{\varphi}{w}{\cdot}$.

%\noindent
%{\em Validity Domain.}
Given a CPSML model $M = (\stsp, \insp, \simu)$, 
and a temporal logic formula $\varphi$,
the \emph{validity domain} of $\varphi$ for model $M$ is the subset of
$\insp$ for which traces of $M$ satisfies $\varphi$.
We denote the validity domain by $\insp_{\varphi}$; the remaining
set of inputs $\insp \setminus \insp_{\varphi}$ is denoted by
$\insp_{\neg\varphi}$.
{\color{blue} Rev 1.2: Note that there are no limitations on the dimensionality of a validity domain. It can potentially characterize
single initial conditions as well as input traces.}
Simulation-based verification tools (such as~\cite{donze2010breach})
can approximately compute validity domains via sampling-based methods.

{\color{blue} Rev 1.1: The peculiarity of the quantitative semantics is that it establishes how robustly a formula is satisfied.
Intuitively, the quantitative evaluation of a formula provides a real value representing
the distance to satisfaction or violation.
The  quantitative  and  qualitative  semantics  are  connected. Specifically, it holds that 
$\rob{\varphi}{w}{t} > 0$ if and only if $w,t \models \varphi$~\cite{donze2013efficient}.}


